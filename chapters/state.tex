\chapter{Estudi de la situació actual}
\section{Protocols i programaris existents}
Els protocols de seguretat poden proveir alguns dels serveis de seguretat aquí explicats:
\begin{itemize}
\item \textbf{Xifrar el tràfic} de manera que no pugi ser llegit per ningú més que les parts a les que s'està dirigint.
\item \textbf{Validació de la integritat} per assegurar que el tràfic no ha estat modificat durant el seu recorregut.
\item \textbf{Autenticar els extrems} per assegurar que el tràfic prové d'un extrem de confiança.
\item \textbf{Evitar el repudi} de l'altre extrem per tal de que no pugui negar haver enviat l'informació.
\item \textbf{Anti-repetició} per protegir-se contra la repetició malintencionada de paquets.
\end{itemize}

A continuació s'analitzaran alguns protocols i programaris existents.
\subsection{IPsec}
\keyword{IPsec}{Internet Protocol security} és d'ús opcional en IPv4 i serà una obligatori en IPv6. IPsec va ser creat per proporcionar seguretat en mode de transport (extrem a extrem) del tràfic de paquets, en el que els ordenadors dels extrems finals realitzen el processat de seguretat, o en mode túnel (porta a porta) en el que la seguretat del tràfic de paquets es proporcionada a vàries màquines (inclús a tota la LAN) per un únic node.

L'ús principal d'IPsec és el de crear VPNs en qualsevol els dos modes. Però les implicacions de seguretat són bastant diferents entre els dos modes d'operació.

La seguretat de comunicacions extrem a extrem a escala Internet és va desenvolupar més tard i utilitza l'infraestructura de clau pública universal \keyword{DNSSEC}{Domain Name System Security Extensions}.

IPsec va ser introduït per proporcionar serveis de seguretat tals com: xifrar el tràfic, validació d'integritat, autenticar als extrems i anti-repetició.

La majoria de implementacions d'aquest protocol té problemes de compatibilitat amb les altres implementacions del mateix protocol degut a que cada fabricant interpreta el \keyword{RFC}{Request for Comments (IETF: Internet Engineering Task Force)} i peculiaritza el protocol a la seva mida. L'ús de DNSSSEC per l'intercanvi de claus no es troba en gaires implementacions.

Així doncs depenent del nivell en el que actuï l'IPsec, aquest estarà treballant en mode transport o mode túnel.
\subsubsection{Mode transport}
En mode transport, només es xifra i/o autentica el \emph{payload} (la carga útil) del paquet IP. Per tant al no tocar les capçaleres IP no afecta al enrutament, però si s'utilitza l'\keyword{AH}{Authentication Header} les direccions IP no poden ser traduïdes (com fa per exemple un \keyword{NAT}{Network Address Translator}) ja que això faria que el valor del \emph{hash} (resum) no coincidís. Aquest mode s'utilitza en comunicacions ordinador a ordinador.

Una forma d'encapsular missatges IPsec per travessar NATs s'ha definit en un un RFC com a mecanisme \keyword{NAT-T}{NAT Traversal in the IKE}.
\subsubsection{Mode túnel}
En el mode túnel, es xifra i/o autentica tot el paquet IP sencer (capçalera inclusiva). Aquesta informació s'encapsula dins del \emph{payload} d'un nou paquet IP per tal de que pugui ser enrutat. Aquest mode s'utilitza en comunicacions xarxa a xarxa, per exemple per crear VPNs a traves d'Internet.

\subsection{OpenVPN}
OpenVPN es un software de VPN a nivell de aplicació que fa ús del \emph{driver} genèric i multiplataforma \okeyword{TUN}/\okeyword{TAP}. És una aplicació centralitzada tal com mostra la figura \ref{F:vpn-centralized}.

\begin{figure}[htb]
\centering
\includegraphics[height=0.5\textwidth]{images/vpn-centralized}
\caption{Diagrama de funcionament d'una VPN centralitzada}
\label{F:vpn-centralized}
\end{figure}

La VPN pot operar en dos modes:
\begin{itemize}
\item \textbf{Mode pont Ethernet}
Simula una targeta de xarxa Ethernet i crea una VPN que treballa amb trames de capa 2 del \keywords{OSI}{model OSI}{Open Systems Interconnection Basic Reference Model}.
\item \textbf{Mode túnel IP}
Simula una targeta de xarxa punt-a-punt i crea una VPN que treballa amb trames de capa 3 del model OSI.
\end{itemize}
\subsection{TincVPN}
\begin{figure}[htb]
\centering
\includegraphics[height=0.5\textwidth]{images/vpn-meshed}
\caption{Diagrama de funcionament d'una VPN mallada}
\label{F:vpn-meshed}
\end{figure}

\subsection{Hamachi}
\subsection{Wippien}
\subsection{ELA VPN}
\keyword{ELA}{Everywhere Local Area network} es un software de VPN a nivell de aplicació.
Utilitza una topologia de nodes mallada connectada tant amb túnels TCP com UDP.
D'aquesta VPN no s'ha trobat cap implementació, el \emph{paper} descriptiu està adjuntat als annexos.
\section{Comparativa}
