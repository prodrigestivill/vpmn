\chapter{Tests i resultats}
\section{Tests}
En aquest capítol s'exposen els resultats dels tests realitzats per comparar l'aplicació creada amb el programari existent.
En tots els programes s'han utilitzat certificats de 1024 bits i s'ha activat la compressió que permetés l'aplicació; el motiu d'activar la compressió és que la llibreria OpenSSL utilitzada per la creació d'aquest programa no permet desactivar la compressió en temps d'execució.
L'escenari dels tests consta de una xarxa Gigabit Ethernet on hi estan connectats dos ordenadors:
\begin{enumerate}
\item Debian GNU/Linux (PC amd64)
\begin{itemize}
\item AMD Athlon 64 X2 Dual Core Processor BE-2400 (2.3GHz, 1024KB cache)
\item nForce Gigabit Ethernet CK804
\end{itemize}
\item Debian GNU/Linux (PC i686)
\begin{itemize}
\item Intel Pentium M Centrino Processor (1.70GHz, 2048KB cache)
\item Realtek Gigabit Ethernet RTL-8169
\end{itemize}
\end{enumerate}
%TODO:
%Afegir retards als tests
%Tests de Compressio
\subsection{Eficiència}
\begin{table}[htb]
\begin{center}
\begin{tabular}{|c|r|r|}
\multicolumn{1}{c}{} & \multicolumn{2}{|c|}{Mida (bytes)} \\ \hline
Aplicació & Inicialització & Ping \\ \hline \hline
\tt directe & 112 & 84 \\ \hline
OpenVPN & 11876 & 153 \\ \hline
TincVPN & 2569 & 136 \\ \hline
\bf VPMN & 4641 & 169 \\ \hline
\end{tabular}
\end{center}
\begin{center}
\caption{Eficiència}
\label{T:efi}
\end{center}
\end{table}
%Es a nivell IP i per tant s'ha restat 14bytes per paquet ethernet.
%TincVPN no intercanvia els certificats pq ja ho tenen fet. 830*2=1660
%El directe s'ha contat el ARP com a inicialtizació i a la resta de programes no.

\subsection{Latència}
\begin{table}[htb]
\begin{center}
\begin{tabular}{|c|c|c|c|c|c|}
\multicolumn{2}{c}{} & \multicolumn{4}{|c|}{RTT (ms)} \\ \hline
Aplicació & Pèrdues & Mínim & Mitjana & Màxim & D.Estàndard \\ \hline \hline
\tt directe & 0\% & 0.047 & 0.050 & 0.074 & 0.006 \\ \hline
OpenVPN & 0\% & 0.199 & 0.479 & 5.333 & 1.080 \\ \hline
TincVPN & 0\% & 0.136 & 0.185 & 0.911 & 0.092 \\ \hline
\bf VPMN & 0\% & 0.228 & 0.310 & 0.415 & 0.034 \\ \hline
\end{tabular}
\end{center}
\begin{center}
\caption{Latència}
\label{T:lat}
\end{center}
\end{table}
%ping -i 2 -c 100

\subsection{Taxa màxima de transferència}
\begin{table}[htb]
\begin{center}
\begin{tabular}{|c|c|c|r|}
\multicolumn{1}{c}{} & \multicolumn{3}{|c|}{Màxim (Mbps)} \\ \hline
Aplicació & TCP & UDP & MGEN \\ \hline \hline
\tt directe & 690.49 & 960.63 & 116.03 \\ \hline
OpenVPN & 185.01 & 613.67 & 78.52 \\ \hline
TincVPN & 133.26 & 286.29 & 78.29 \\ \hline
\bf VPMN & 188.09 & --- & 96.17 \\ \hline
\end{tabular}
\end{center}
\begin{center}
\caption{Taxa màxima de transferència}
\label{T:tax}
\end{center}
\end{table}
%netperf per TCP
\begin{table}[htb]
\begin{center}
\begin{minipage}[htb]{0.6\linewidth}
\footnotesize
\begin{verbatim}
TXBUFFER 1000
0.0 ON  1 UDP DST 10.0.0.1/5000 PERIODIC [90000.0 996]
10.0 OFF 1
\end{verbatim}
\end{minipage}
\caption{Configuració MGEN}
\label{T:mgencfg}
\end{center}
\end{table}

\begin{minipage}[htb]{\linewidth}
\begin{equation}\label{E:mgen}
r=P\cdot\frac{(996+28)\cdot8}{10}
\end{equation}
\centering
{\scriptsize
r: Tassa de transmissió. 
P: Número de paquets rebuts. 
}\\
Equació \ref{E:mgen}: Càlcul de la tassa de transmissió del MGEN.
\end{minipage}

%Proves amb varios nodes
%\subsection{Capacitat}
%\subsection{Càrrega}
%\subsection{Escalabilitat}
\section{Anàlisi de resultats}
