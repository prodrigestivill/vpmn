
\documentclass[twocolumn]{article} % default is 10 pt
\usepackage{graphicx} % needed for including graphics e.g. EPS, PS

\long\def\comment#1{}

% uncomment if don't want page numbers
% \pagestyle{empty}

%set dimensions of columns, gap between columns, and paragraph indent 
\setlength{\textheight}{8.75in}
\setlength{\columnsep}{0.375in}
\setlength{\textwidth}{6.8in}
\setlength{\topmargin}{0.0625in}
\setlength{\headheight}{0.0in}
\setlength{\headsep}{0.0in}
\setlength{\oddsidemargin}{-.19in}
\setlength{\parindent}{0pt}
\setlength{\parskip}{0.12in}
\makeatletter
\def\@normalsize{\@setsize\normalsize{10pt}\xpt\@xpt
\abovedisplayskip 10pt plus2pt minus5pt\belowdisplayskip 
\abovedisplayskip \abovedisplayshortskip \z@ 
plus3pt\belowdisplayshortskip 6pt plus3pt 
minus3pt\let\@listi\@listI}

%need an 11 pt font size for subsection and abstract headings 
\def\subsize{\@setsize\subsize{12pt}\xipt\@xipt}
%make section titles bold and 12 point, 2 blank lines before, 1 after
\def\section{\@startsection {section}{1}{\z@}{1.0ex plus
1ex minus .2ex}{.2ex plus .2ex}{\large\bf}}
%make subsection titles bold and 11 point, 1 blank line before, 1 after
\def\subsection{\@startsection 
   {subsection}{2}{\z@}{.2ex plus 1ex} {.2ex plus .2ex}{\subsize\bf}}
\makeatother

\begin{document}

% don't want date printed
\date{}

% >>>>>>>>>>>>>>>>>>>>>>>  Put your title here <<<<<<<<<<<<<<<<<<<<<<<<
% make title bold and 14 pt font (Latex default is non-bold, 16pt) 
\title{\Large {\bf Replace This Line With Your Title }}

% >>>>>>>>>>>>>>>>>>>>>>> Author's Name, Thanks or Affliation <<<<<<<<
\author{John H. Smith
 \thanks{Text in here gets placed in a footnote section.  Typically you
 thank sponsors here or perhaps add your address e.g. 
 Drexel University Mechanical Engineering, Robotics 
 \& Machine Vision Lab, Philadelphia PA USA 19104 
 Tel/Fax: 215-895-6396/1478 Email: paul@coe.drexel.edu
 }
}

\maketitle
\thispagestyle{empty}

\subsection*{\centering Abstract}

% >>>>>>>>>>>>>>>>>>>>>>>>> Keywords and Abstract <<<<<<<<<<<<<<<<<<<<<
% Replace with your own keywords and abstract.  Text will be in italics
{\em Keywords: 
visual-servoing, tracking, biomimetic, redundancy, degrees-of-freedom

Motion tracking and object recognition often use cameras
that are mounted in motion platforms like pan-tilt units,
linear tables and even robots.  Tracking can be
automated by visually servoing the platform's degrees-of-freedom (DOF)
thus keeping the camera's point-of-view directed at the target.
Tracking quick moving targets often demands faster bandwidth platforms.  
However biology suggests a redundant approach where DOF, like the eye and
head, cooperate to direct vision systems and overcome joint limits.  This
paper illustrates the effectiveness of this concept using a 
robot-mounted camera.
}

% >>>>>>>>>>>>>>>>>>>>>> START OF YOUR PAPER <<<<<<<<<<<<<<<<<<<<<<<<<<<<<<
% Typically paper starts of with an Introduction header.  Replace text in
% the french braces if you see fit.  I also typically name the label the
% same as the section

\section{Introduction}
\label{Introduction}

Visual coverage of large areas often demands mounting cameras on motion
platforms like booms, gantries, planetary rovers, industrial robots, aircraft 
and submersibles.  Previous work in visually servoing a 5-DOF hybrid gantry robot 
demonstrated that a coupled coordination of redundant DOF, called {\it partitioning}, 
can track faster moving subjects \cite{OhTRA2001}. 

Section~\ref{System Description} describes our tracking interests with a 
camera-carrying gantry along with brief highlights of previous gain-tuning results.  
Section~\ref{Cost Function} defines the tracking task and DOF coordination using a 
cost function to optimally tune gains and develops the necessary controller. 

\section{System Description}
\label{System Description}

Our particular tracking interests are in visually monitoring tools, workpieces 
or people moving in a large ($3.6\times6.4\times1.0$ $m^3$) manufacturing workcell,
Figure~\ref{overView}. We thus built a ceiling-mounted Cartesian gantry and attached a 2-DOF 
pan-tilt unit (PTU) to its end-effector.  The gantry's large inertias limit camera translational
velocities but allow the camera to be positioned anywhere in the workcell.  By 
comparison, the PTU only servos a light-weight camera and thus camera orientations are 
quick but its joint limits constrain what fields-of-view can be obtained.  The net 
effect is a hybrid robot characterized by DOF dynamics, kinematics and redundancy.
The PTU can point the camera at the subject and the gantry can maneuver the camera
to maintain a desired camera-to-subject depth or achieve line-of-sight if subject
occlusion occurs.

% >>>>>>>>>>>>>>>>>>>>>>>>>>> Example of including a figure <<<<<<<<<<<<<<<<<<<<<<

Here, 2 copies of the same figure are included and lie side-by-side.  Latex only 
accepts EPS or PS graphics formats.  The width can be changed from 3 inches
if you wish.  For caption, change the text between the french braces.
Lastly, give the figure a label so you can reference it later
You don't have to worry about where the figure is placed.  Latex does the
placement for you by considering where in the text the figure is referenced.

\begin{figure*}
\centerline{
\mbox{\includegraphics[width=3.00in]{assemblyWorkcellOverview1_0.eps}}
\mbox{\includegraphics[width=3.00in]{assemblyWorkcellOverview1_0.eps}}
}
\caption{Replace text here with your desired caption.}
\label{overView}
\end{figure*}

% >>>>>>>>> Add a subsection if you want <<<<<<<<<<<<<<<<<<<<<<<<<<<<<<<<

\subsection{Cost Function}
\label{Cost Function}

Notice that whatever number the parent section is, the child subsection will be 
automatically numbered.

% >>>>>>>>>>>>>>>>>>>> Add an equation <<<<<<<<<<<<<<<<<<<<<<<<<<<<<<<<<<

Equation numbers are automatically generated.  Label allows easy 
referencing throughout the paper

\begin{equation}
{\bf X}[k+1]=A{\bf X}[k]+B{\bf u}[k]
\label{stateSpaceForm1}
\end{equation}

% >>>>>>>>>>>>>>>>>>> Add an unnumbered equation <<<<<<<<<<<<<<<<<<<<<<<<

You can also add an unnumbered equation as follows

$$
\theta_c[k+1]=\theta_c[k]+Tu_p[k]
$$

\section{Conclusions and Future Work}

To demonstrate Latex's ability to easily reference sections, equations
figures and bibliographical references consider the following.  
Section\ref{Introduction} described the general problems motivating
the research.

% >>>>>>>>>>>>>>>>>>>>>> Bibliography <<<<<<<<<<<<<<<<<<<<<<<<<<<<<<<<<<<<<
% A label is give for each bibitem.  Your paper references this label

\begin{thebibliography}{99}

% >>>>>>>>> Book examples <<<<<<<<<
\bibitem{CarpenterBOOK} Carpenter, R.H.S., {\it Movements of the Eyes},
 2nd Edition, Pion Publishing, 1988.

\bibitem{FranklinBOOK} Franklin, G.F., Powel, J.D., Workman, M.L.,
{\it Digital Control of Dynamic Systems}, Second Edition,
Addison-Wesley, 1990.

% >>>>>>>>> Conference Proceedings Example <<<<<<<<<
\bibitem{OhICRA1998} Oh, P.Y., Allen, P.K., ``Design a Partitioned
 Visual Feedback Controller,'' {\it IEEE Int Conf Robotics
 \& Automation}, Leuven, Belgium, pp. 1360-1365 5/98

% >>>>>>>>> Journal Example <<<<<<<<<<<<<<<<<<<<<<<<
\bibitem{OhTRA2001} Oh, P.Y., Allen, P.K., ``Visual Servoing
 by Partitioning Degrees of Freedom,'' {\it IEEE Trans on 
 Robotics \& Automation}, V17, N1, pp. 1-17, 2/01

\end{thebibliography}
\end{document}

