\documentclass{beamer}
\mode<presentation>
{
  %\usetheme{Warsaw}
  %\setbeamercovered{transparent}
  % or ...
  \usetheme{Darmstadt}
  \usefonttheme[onlylarge]{structurebold}
  \setbeamerfont*{frametitle}{size=\normalsize,series=\bfseries}
  \setbeamertemplate{navigation symbols}{}
}

\usepackage[english]{babel}
% or whatever

\usepackage[utf8]{inputenc}
% or whatever

\usepackage{times}
\usepackage[T1]{fontenc}
% Or whatever. Note that the encoding and the font should match. If T1
% does not look nice, try deleting the line with the fontenc.


\title[VPMN]{Virtual Private Mesh Network} % (optional, use only with long paper titles)

\subtitle{Decentralized VPN application} % (optional)

\author[Pau]{Pau~Rodriguez} % (optional, use only with lots of authors)

\institute[EPSC UPC] % (optional, but mostly needed)
{
  Escola Polit\`ecnica Superior de Castelldefels\\
  Universitat Polit\`ecnica de Catalunya
}
% - Use the \inst command only if there are several affiliations.
% - Keep it simple, no one is interested in your street address.

\date[SCG08] % (optional)
{Summer Camp Garrotxa 2008}

\subject{Summer Camp Garrotxa 2008 - Talk}
% This is only inserted into the PDF information catalog. Can be left
% out. 

% If you have a file called "university-logo-filename.xxx", where xxx
% is a graphic format that can be processed by latex or pdflatex,
% resp., then you can add a logo as follows:

% \pgfdeclareimage[height=0.5cm]{university-logo}{university-logo-filename}
% \logo{\pgfuseimage{university-logo}}

\pgfdeclareimage[height=0.75em]{scg08-logo}{scg08/logo}
\logo{\pgfuseimage{scg08-logo}}

% Delete this, if you do not want the table of contents to pop up at
% the beginning of each section:
\AtBeginSection[]
{
  \begin{frame}<beamer>{Outline}
    \tableofcontents[currentsection]
  \end{frame}
}

% If you wish to uncover everything in a step-wise fashion, uncomment
% the following command: 

%\beamerdefaultoverlayspecification{<+->}

\newcommand*{\comparativeframe}[4]{
\begin{frame}{#1}
	#2
	\begin{block}{Pros}
		\begin{itemize}
		#3
		\end{itemize}
	\end{block}
	\begin{block}{Cons}
		\begin{itemize}
		#4
		\end{itemize}
	\end{block}
\end{frame}
}

\begin{document}
\pdfbookmark[2]{Welcome}{welcome}
\begin{frame}
  \titlepage
\end{frame}

\begin{frame}{Outline}
  \tableofcontents
  % You might wish to add the option [pausesections]
\end{frame}

% However, the talk length of between 15min and 45min and
% the theme suggest that you stick to the following rules:

% - Exactly two or three sections (other than the summary).
% - At *most* three subsections per section.
% - Talk about 30s to 2min per frame. So there should be between about
%   15 and 30 frames, all told.

\section{Introduction}
\subsection*{Status}
\begin{frame}{Internet}
	\begin{itemize}
	\item Prova
	\end{itemize}
\end{frame}
\begin{frame}{VPN}
	\begin{center}
	\includegraphics[height=8em]{images/vpn}
	\end{center}
	\begin{itemize}
	\item Prova
	\end{itemize}
\end{frame}
\begin{frame}{Security}
	\begin{itemize}
	\item \textbf{Encrypt the traffic} de manera que no pugi ser llegit per ningú més que les parts a les que s'està dirigint.
	\item \textbf{Validació de la integritat} per assegurar que el tràfic no ha estat modificat durant el seu recorregut.
	\item \textbf{Autenticar els extrems} per assegurar que el tràfic prové d'un extrem de confiança.
	\item \textbf{Anti-repetició} per protegir-se contra la repetició malintencionada de paquets.
	\item \textbf{Evitar el repudi} de l'altre extrem per tal de que no pugui negar haver enviat l'informació.
	\end{itemize}
\end{frame}
\subsection*{Objectives}
\begin{frame}{Project Aims}
	\begin{center}
	\includegraphics[height=10em]{images/vpn-meshed}
	\end{center}
	\begin{itemize}
	\item A dynamic VPN softwareThe \alert{first main message} of your talk in one or two lines.
	\item The \alert{second main message} of your talk in one or two lines.
	\item Perhaps a \alert{third message}, but not more than that.
	\end{itemize}
\end{frame}

\section{State of Art}
\subsection{IPsec}
\comparativeframe{IPsec vs VPMN}{}
{
\item Prova
}{
\item Prova2
}
\subsection{OpenVPN}
\begin{frame}{OpenVPN summary}
	\begin{center}
	\includegraphics[height=10em]{images/vpn-centralized}
	\end{center}
\end{frame}
\comparativeframe{OpenVPN vs VPMN}{}{
\item Prova
}{
\item Prova2
}
\subsection{Tinc VPN}
\comparativeframe{Tinc VPN vs VPMN}{}
{
\item Prova
}{
\item Prova2
}

\section{Internals}
\subsection{Overview}
\begin{frame}{Application}
	\begin{itemize}
	\item Prova
	\end{itemize}
\end{frame}
\subsection{Security}
\begin{frame}{Security summary}
	\begin{itemize}
	\item Prova
	\end{itemize}
\end{frame}
\begin{frame}{DTLS}
	\begin{itemize}
	\item Prova
	\end{itemize}
\end{frame}
\begin{frame}{Packets}
	\begin{itemize}
	\item Prova
	\end{itemize}
\end{frame}
\subsection{Architecture}
\begin{frame}{UDP Server}
	\begin{itemize}
	\item Prova
	\end{itemize}
\end{frame}
\begin{frame}{TUN Server}
	\begin{itemize}
	\item Prova
	\end{itemize}
\end{frame}
\subsection{Benefits}
\begin{frame}{What's done?}
	\begin{itemize}
	\item Prova
	\end{itemize}
\end{frame}
%\section{Tests}

\section*{Summary}
%\subsection*{Summary}
\begin{frame}{Summary}
	\begin{itemize}
	\item \alert{Prova}
	\end{itemize}
\end{frame}
\end{document}
